\documentclass[11pt,a4paper]{article}
\usepackage[utf8]{inputenc}
\usepackage{amsmath}
\usepackage{amsfonts}
\usepackage{amssymb}
\usepackage{graphicx}
\usepackage{hyperref}
\usepackage{authblk}
\usepackage{german}
\usepackage{fancyhdr}
\usepackage{thesis}



% Changes header
\pagestyle{fancy}
\lhead{Stefan Pitirut, Andreas Vogt}
\rhead{DEZSYS}

\title{Load Balancing\\ DEZSYS-05}
\author{Stefan Pitirut \& Andreas Vogt} 

\renewcommand\Authands{ and }

\date{\today{}, Vienna}

\begin{document}
\maketitle

\newpage
\tableofcontents

\newpage
\section{Aufgabenstellung}
Es soll ein Load Balancer mit mindestens 2 unterschiedlichen Load-Balancing Methoden (jeweils 7 Punkte) implementiert werden (ähnlich dem PI Beispiel [1]; Lösung zum Teil veraltet [2]). Eine Kombination von mehreren Methoden ist möglich. Die Berechnung bzw. das Service ist frei wählbar!

Folgende Load Balancing Methoden stehen zur Auswahl:
\begin{itemize}
	\item Weighted Round-Round
	\item Least Connection
	\item Least Connected Slow- Start Time
	\item Weighted Least Connection
	\item Agent Based Adaptive Balancing / Server Probes
\end{itemize}

Um die Komplexität zu steigern, soll zusätzlich eine "Session Persistence" (2 Punkte) implementiert werden.

Tests

Die Tests sollen so aufgebaut sein, dass in der Gruppe jedes Mitglied mehrere Server fahren und ein Gruppenmitglied mehrere Anfragen an den Load Balancer stellen. Für die Abnahme wird empfohlen, dass jeder Server eine Ausgabe mit entsprechenden Informationen ausgibt, damit die Verteilung der Anfragen demonstriert werden kann.


Modalitäten

Gruppenarbeit: 2 Personen
Abgabe: Protokoll mit Designüberlegungen / Umsetzung / Testszenarien, Sourcecode (mit allen notwendigen Bibliotheken), Java-Doc, Jar


Viel Erfolg!


Quellen

[1] "Praktische Arbeit 2 zur Vorlesung 'Verteilte Systeme' ETH Zürich, SS 2002", Prof.Dr.B.Plattner, übernommen von Prof.Dr.F.Mattern (http://www.tik.ee.ethz.ch/tik/education/lectures/VS/SS02/Praktikum/aufgabe2.pdf)
[2] http://www.tik.ee.ethz.ch/education/lectures/VS/SS02/Praktikum/loesung2.zip

\section{Designüberlegung}
Unser erster Schritt war es zwei geeignete Load-Balancing Methoden auszuwählen. Zur Auswahl standen
\begin{itemize}
	\item Weighted Round-Round
	\item Least Connection
	\item Least Connected Slow- Start Time
	\item Weighted Least Connection
	\item Agent Based Adaptive Balancing / Server Probes
\end{itemize}
Nachdem wir jede Methode angeschaut hatten, sind wir zu dem Entschluss gekommen, dass wir mit Weighted Round-Robin und 
Least Connection gut zurecht kommen und optimal für die Aufgabe sind
\bibliography{Loead Balancer}
\bibliographystyle{plain}


\end{document}